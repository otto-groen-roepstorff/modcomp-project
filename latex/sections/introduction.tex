\section{Introduction}
We want to explore the mechanisms behind the joint behaviour of neurons attending to visual stimuli by analyzing neuron spike counts using Hidden Markov Models.

Neuron spikes are instantaneous events that show activity in a neuron, and the number of neuron spikes are interpreted as an indicator of neuron activity. The more spikes we register in a time span, the more active the neuron is in that time span. We need to count the number of spikes within time periods to compare how different neurons react to different stimuli. We assume that different stimuli induce different levels of neuron activity.

The underlying model structure of reaction to simuli is as follows. A collection of neurons may react to stimuli in uncoupled attention (parallel processing) or coupled attention (serial processing), which determines the probability of a single neuron attending to visual stimuli. If the brain is in the state of uncoupled attention, we would expect every neuron to independently decide which stimulus to attend, while coupled attention would indicate it is more likely for all neurons to attend the same stimulus at the same time. Based on the neuron spike counts for observed data we want to infer if the brain in is parallel processing or serial processing.

If there is no difference in brain activity over time when shown two different stimuli, it would indicate that the brain always parallel processes since the two stimuli would be attended equally. If there are big differences in the average neuron count across the time span and if we can see some kind of pattern, we would expect the brain to be more more in the state of serial processing.

\subsection{Setting}
The goal is to explore the data of $n$ neurons observed in some time period subdivided into $T$ intervals of the form $[t_1, t_i+1)$. We will as abuse of notation say that time $j$ is the same as time interval $[t_j, t_j+1)$ for $j = 1,\ldots, T-1$.

Let $N_{i,j}$ be neuron $i$ at time $j$, and denote the two stimuli  as $S_0$ and $S_1$.



neurons attend stimuli as serial processing or parallel processing.

The generative variables are

\begin{itemize}
    \item $Z_{i,j}:$ A hidden indicator of the form $ \mathrm I (\text{neuron } N_{i,j} \text{ attends stimuli } S_1)$
    \item $X_{i,j}:$ Observed number of spikes measured in neuron $N_{i,j}$
    \item $C_i$: Hidden indicator that takes the value 
        \subitem $0$: Majority of neurons are expected to attend stimulus 0  at time (Serial processing)
        \subitem $1$: Majority of neurons are expected to attend stimulus 1 (Serial processing)
        \subitem $2$: Each individual neuron has $50\%$ chance of observing either $S_0$ or $S_1$(Parallel processing)
\end{itemize}

\subsection{Assumptions}
The brain cannot go from solely attending $S_0$ to attending $S_1$ that is $P(C_t =1 \mid C_{t-1} = 0) = P(C_t = 0 \mid C_{t-1} = 1) = 0$. We always need to go from the serial state to the parallel stat and then possibly to the serial state again.



